\documentclass[main.tex]{subfiles}

\onlyinsubfile{\usepackage[utf8]{inputenc}

\usepackage{geometry}   
\geometry{
 a4paper,
 left = 20mm,
 right = 20mm,
 top = 20mm,
 bottom = 20mm,
 margin = 1in,
 heightrounded,
 headheight = 110pt
}

\usepackage{amsmath, amsthm, amssymb}
\swapnumbers % Switch number/label style
% https://tex.stackexchange.com/questions/32100/what-does-each-ams-package-do

\usepackage{mathtools} 
% https://tex.stackexchange.com/questions/46308/oversized-underbraces-label-causes-unwanted-spacing
% used with \mathclap, \mathllap, \mathrlap, fixes spacing, also use \strut, example: \left( x-1 \right) \underbrace{ \left( ? \right) }_{ \mathclap{ \text{ \strut what is this? } } }

%\usepackage{bbm} 
% http://tug.ctan.org/macros/latex/contrib/bbm/bbm.pdf

%\usepackage{enumitem}
% http://www.texnia.com/archive/enumitem.pdf


%\usepackage[hang]{footmisc} 
%\setlength\footnotemargin{10pt}
% https://tex.stackexchange.com/questions/126877/how-can-i-align-a-multiple-line-footnote-text-right-to-the-footnote-mark

\usepackage{mathrsfs} % \mathscr{ABCDEFGHIJKLMNOPQRSTUVWXYZ}
% https://www.stat.colostate.edu/~vollmer/pdfs/typesetting-script.pdf
% https://www.stat.colostate.edu/~vollmer/pdfs/typesetting-script.pdf

%\usepackage{marvosym} % \Lightning in text-mode
% http://texdoc.net/texmf-dist/doc/fonts/marvosym/marvodoc.pdf


% Customize page header and footer ------------------
\usepackage{fancyhdr} 
% https://en.wikibooks.org/wiki/LaTeX/Customizing_Page_Headers_and_Footers#Customizing_with_fancyhdr

\usepackage{lastpage}
% https://tex.stackexchange.com/questions/227/how-can-i-add-page-of-on-my-document

% setting...
\pagestyle{fancy}
\lhead{~}
\chead{\nouppercase{\leftmark}} % casing
\rhead{~} 
\lfoot{Kristoffer Brix}
\cfoot{~}
\rfoot{Page \thepage}
%\rfoot{\thepage\ of \pageref{LastPage}}
\renewcommand{\headrulewidth}{0.4pt}
\renewcommand{\footrulewidth}{0.4pt}
% ...

\fancypagestyle{plain}{
    \fancyhead{}
    \renewcommand{\headrulewidth}{Opt}
}
% https://tex.stackexchange.com/questions/86708/how-do-i-make-a-header-for-whole-document
% FINISH: Customize page header and footer ----------

% Misc. settings --------------
\renewcommand{\baselinestretch}{1.15} %spacing between lines
\usepackage[parfill]{parskip} %removes indentation
\swapnumbers
\numberwithin{equation}{section}
\numberwithin{figure}{section}
\addtolength{\jot}{4.5pt} % more spacing between math... default is 3pt
% ----------------------------

% Caption settings -------------------------------
\usepackage[font=small, labelfont=bf]{caption}

\usepackage[hypertexnames=false]{hyperref}  
% https://tex.stackexchange.com/questions/823/remove-ugly-borders-around-clickable-cross-references-and-hyperlinks

\usepackage{xcolor}
\hypersetup{
    colorlinks,
    linkcolor={red!50!black},
    citecolor={blue!50!black},
    urlcolor={blue!80!black}
} 

% FINSISH caption settings ----------------------


% https://www.tug.org/TUGboat/tb27-2/tb87syropoulos.pdf
%\usepackage{phaistos} % \PHdove \PHtunny

% https://tex.stackexchange.com/questions/191572/beginframed-with-background-color
\usepackage{mdframed}

% for \FloatBarrier
%\usepackage{placeins}

%\usepackage{printlen} 
% \printlength\textwidth \printlength\textheight https://tex.stackexchange.com/questions/39383/determine-text-width

%%%%%%%%%%%%%%%%%%%%%%%%%%%%%%%%%%%%%%% REMOVE WHEN DONE
%\iffalse % HERE
\usepackage{seqsplit}
\usepackage[color]{showkeys}
\definecolor{refkey}{rgb}{0, 0, 1}
\definecolor{labelkey}{rgb}{0, 0, 1}

\renewcommand*\showkeyslabelformat[1]{\normalfont\large\ttfamily(#1)}  

\usepackage{xstring}

\renewcommand*\showkeyslabelformat[1]{%
\noexpandarg%
% instead of \textvisiblespace you can also put in ~
% if you want to keep a plain space at space characters
\StrSubstitute{\(\{\)#1\(\}\)}{ }{\textvisiblespace}[\TEMP]%
\parbox[t]{\marginparwidth}{\raggedright\normalfont\small\ttfamily\expandafter\seqsplit\expandafter{\TEMP}}}
%https://tex.stackexchange.com/questions/148579/tweak-showlabels-showkeys-wrap-the-label
%\fi %HERE 
%%%%%%%%%%%%%%%%%%%%%%%%%%%%%%%%%%%%%%% REMOVE WHEN DONE
}     

\onlyinsubfile{\usepackage{xr}}
% https://www.overleaf.com/learn/how-to/Cross_referencing_with_the_xr_package_in_Overleaf

\onlyinsubfile{     
 \geometry{
   reset,
   a4paper,
   textwidth=5cm,textheight=5cm,
   % exact values will vary by template
   paperwidth=453pt+0.8cm,
   paperheight=706pt+2cm, 
   % above values found using \usepackage{printlen} \printlength\textwidth \printlength\textheight https://tex.stackexchange.com/questions/39383/determine-text-width
   text={453pt,706pt}, % without this line, errors occur
   %marginparwidth = 2cm,
   marginparsep = -2.5cm
 }
} 

\begin{document}
 
\section{Problem 641}

\begin{mdframed}[backgroundcolor = lightgray!10]
Consider a row of $n$ dice all showing $1$.

First turn every second die, $(2, 4, 6, \dots)$, so that the number showing is increased by $1$. Then turn every third die. The sixth die will now show a $3$. Then turn every fourth die and so on until every $n$'th die (only the last die) is turned. If the die to be turned is showing a 6 then it is changed to show a 1.

Let $f(n)$ be the number of dice that are showing a $1$ when the process finishes. You are given $f(100) = 2$ and $f(10^8) = 69$.


Find $f(10^{36})$.
\end{mdframed}

Upon inspection, one sees that the problem is asking us to compute
%
\begin{align}
 \# \{ k \in \{1, \dots, n \} \mid
 \sigma(k) \text{ mod } 6 = 1\},
\end{align}
%
where $\sigma: \mathbb N \to \mathbb N$ is the function that counts how many divisors a given integer has, e.g. $\sigma(12) = 6$ and $\sigma(p) = 2$, where $p$ is a prime number. 

In words, we are tasked to find numbers satisfying 
%
\begin{align}
 \sigma(k) = 1, 7, 13, 19, \dots
\end{align}

Since the above numbers are all odd, we may search through the squares:
%
\begin{align}
 \# \{ k = \widetilde k^2 \in \{1, \dots, n \} \mid
 \sigma(k) \text{ mod } 6 = 1\},
\end{align}
%
A quick and brute-force approach easily yields the results given in the table below. 

Upon inspecting the table, we see that the $n$ of interest has representation
%
\begin{align}
 n = a^6 b^4,
 \qquad b = \prod_{i=1}^{2m} p_i,
 \qquad a \in \mathbb N,
 \qquad m \in \mathbb N,
 \qquad p_i\text{'s are distinct is primes}.
\end{align}
%
Most of the values in the table fit the representation above. A few needs to adjusted slightly, e.g. $n = 85525504 = 2^{10} \cdot 17^4 = 2^6 \cdot (2 \cdot 17)^4$. Note that the numbers whose prime factorization contains an even number of distinct primes are those numbers whose M\"{o}bius value is equal to one. 

For each value of $b$, there are
%
\begin{align}
 \frac{n^{1/4}}{b^{3/2}}
\end{align}
%
choices for $a$, because $a^6 b^4 \leq n$ must hold. If we set $b=n^{\frac{1}{6}}$ in the above, then we get $1$.

Let $\mu: \mathbb N \to \{-1, 0, 1\}$ be the M\"{o}bius function and let $M_+: \mathbb R_+ \to \mathbb N$, given by
%
\begin{align}
 M_+(x) := \sum_{i=1}^{\lfloor x \rfloor} \mu^2(i),
\end{align}
%
be the function that counts the natural numbers $i \leq \lfloor x \rfloor$ such that $\mu(i) = 1$.

Then
%
\begin{align}
 f(n)
 =
 \sum_{a = 1}^{n^{1/6}}
 M_+ \left(
 \frac{n^{1/4}}{a^{3/2}}
 \right).
\end{align}
%
The values of M\"{o}bius function can easily be computed using a sieve (the Sieve of Eratosthenes works fine), and the rest is trivial. 

\begin{center}
\begin{tabular} {|rr|r}
$f(n)$ & $n$ &  \\ \hline
  $1$ &        $1$ &                      $1^{1}$ \\
  $2$ &       $64$ &                      $2^{6}$ \\
  $3$ &      $729$ &                      $3^{6}$ \\
  $4$ &     $1296$ &            $2^{4}\cdot3^{4}$ \\
  $5$ &     $4096$ &                     $2^{12}$ \\
  $6$ &    $10000$ &            $2^{4}\cdot5^{4}$ \\
  $7$ &    $15625$ &                      $5^{6}$ \\
  $8$ &    $38416$ &            $2^{4}\cdot7^{4}$ \\
  $9$ &    $46656$ &            $2^{6}\cdot3^{6}$ \\
 $10$ &    $50625$ &            $3^{4}\cdot5^{4}$ \\
 $11$ &    $82944$ &           $2^{10}\cdot3^{4}$ \\
 $12$ &   $117649$ &                      $7^{6}$ \\
 $13$ &   $194481$ &            $3^{4}\cdot7^{4}$ \\
 $14$ &   $234256$ &           $2^{4}\cdot11^{4}$ \\
 $15$ &   $262144$ &                     $2^{18}$ \\
 $16$ &   $456976$ &           $2^{4}\cdot13^{4}$ \\
 $17$ &   $531441$ &                     $3^{12}$ \\
 $18$ &   $640000$ &           $2^{10}\cdot5^{4}$ \\
 $19$ &   $944784$ &           $2^{4}\cdot3^{10}$ \\
 $20$ &  $1000000$ &            $2^{6}\cdot5^{6}$ \\
 $21$ &  $1185921$ &           $3^{4}\cdot11^{4}$ \\
 $22$ &  $1336336$ &           $2^{4}\cdot17^{4}$ \\
 $23$ &  $1500625$ &            $5^{4}\cdot7^{4}$ \\
 $24$ &  $1771561$ &                     $11^{6}$ \\
 $25$ &  $2085136$ &           $2^{4}\cdot19^{4}$ \\
 $26$ &  $2313441$ &           $3^{4}\cdot13^{4}$ \\
 $27$ &  $2458624$ &           $2^{10}\cdot7^{4}$ \\
 $28$ &  $2985984$ &           $2^{12}\cdot3^{6}$ \\
 $29$ &  $3240000$ &  $2^{6}\cdot3^{4}\cdot5^{4}$ \\
 $30$ &  $4477456$ &           $2^{4}\cdot23^{4}$ \\
 $31$ &  $4826809$ &                     $13^{6}$ \\
 $32$ &  $5308416$ &           $2^{16}\cdot3^{4}$ \\
 $33$ &  $6765201$ &           $3^{4}\cdot17^{4}$ \\
 $34$ &  $7290000$ &  $2^{4}\cdot3^{6}\cdot5^{4}$ \\
 ~&~&~
\end{tabular}
~~~~~
\begin{tabular} {|rr|r}
$f(n)$ & $n$ &  \\ \hline
 $35$ &  $7529536$ &            $2^{6}\cdot7^{6}$ \\
 $36$ &  $9150625$ &           $5^{4}\cdot11^{4}$ \\
 $37$ & $10556001$ &           $3^{4}\cdot19^{4}$ \\
 $38$ & $11316496$ &           $2^{4}\cdot29^{4}$ \\
 $39$ & $11390625$ &            $3^{6}\cdot5^{6}$ \\
 $40$ & $12446784$ &  $2^{6}\cdot3^{4}\cdot7^{4}$ \\
 $41$ & $14776336$ &           $2^{4}\cdot31^{4}$ \\
 $42$ & $14992384$ &          $2^{10}\cdot11^{4}$ \\
 $43$ & $16777216$ &                     $2^{24}$ \\
 $44$ & $17850625$ &           $5^{4}\cdot13^{4}$ \\
 $45$ & $20250000$ &  $2^{4}\cdot3^{4}\cdot5^{6}$ \\
 $46$ & $22667121$ &           $3^{4}\cdot23^{4}$ \\
 $47$ & $24137569$ &                     $17^{6}$ \\
 $48$ & $28005264$ &  $2^{4}\cdot3^{6}\cdot7^{4}$ \\
 $49$ & $29246464$ &          $2^{10}\cdot13^{4}$ \\
 $50$ & $29986576$ &           $2^{4}\cdot37^{4}$ \\
 $51$ & $34012224$ &           $2^{6}\cdot3^{12}$ \\
 $52$ & $35153041$ &           $7^{4}\cdot11^{4}$ \\
 $53$ & $36905625$ &           $3^{10}\cdot5^{4}$ \\
 $54$ & $40960000$ &           $2^{16}\cdot5^{4}$ \\
 $55$ & $45212176$ &           $2^{4}\cdot41^{4}$ \\
 $56$ & $47045881$ &                     $19^{6}$ \\
 $57$ & $52200625$ &           $5^{4}\cdot17^{4}$ \\
 $58$ & $54700816$ &           $2^{4}\cdot43^{4}$ \\
 $59$ & $57289761$ &           $3^{4}\cdot29^{4}$ \\
 $60$ & $60466176$ &          $2^{10}\cdot3^{10}$ \\
 $61$ & $64000000$ &           $2^{12}\cdot5^{6}$ \\
 $62$ & $68574961$ &           $7^{4}\cdot13^{4}$ \\
 $63$ & $74805201$ &           $3^{4}\cdot31^{4}$ \\
 $64$ & $75898944$ & $2^{6}\cdot3^{4}\cdot11^{4}$ \\
 $65$ & $78074896$ &           $2^{4}\cdot47^{4}$ \\
 $66$ & $81450625$ &           $5^{4}\cdot19^{4}$ \\
 $67$ & $85525504$ &          $2^{10}\cdot17^{4}$ \\
 $68$ & $85766121$ &            $3^{6}\cdot7^{6}$ \\
 $69$ & $96040000$ &  $2^{6}\cdot5^{4}\cdot7^{4}$  
\end{tabular}
\end{center}
                

\end{document}
